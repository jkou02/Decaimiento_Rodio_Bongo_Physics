\documentclass[12pt, a4paper]{report}

\usepackage[spanish]{babel}
\usepackage[T1]{fontenc}
\usepackage[utf8]{inputenc}
\usepackage{csquotes}  % ← AGREGAR ESTA LÍNEA

\usepackage{graphicx}
\graphicspath{{../figuras/}}

\usepackage{amsmath}
\usepackage{geometry}
\geometry{margin=2.5cm}

\usepackage[style=ieee, backend=biber]{biblatex}
\addbibresource{referencias.bib}

\usepackage{float}
\usepackage{hyperref}  % ← ÚTIL para URLs con acentos

%----------------------------------------------------------------------------------
%----------------------------------------------------------------------------------

% Title Page
\title{Reporte de Resultados: Vida Media}
\author{Josuep Simon Turmero Brito}
\date{\today}

\begin{document}
	\maketitle
	\thispagestyle{empty}
	\newpage
	
	\tableofcontents
	\newpage


%----------------------------------------------------------------------------------
%----------------------------------------------------------------------------------

\section{Objetivo del experimento simulado}
Estudiar el tiempo de vida media de un elemento inestable desconocido.

%----------------------------------------------------------------------------------
%----------------------------------------------------------------------------------

\section{Gráfica de resultados}

\begin{figure}[H]
	\centering
	\includegraphics[width=0.8\textwidth]{datos_originales_Rh99.png}
	\caption{Curva de decaimiento radiactivo del elemento inestable desconocido usando datos simulados.}
	\label{fig:img1}
\end{figure}

\begin{figure}[H]
	\centering
	\includegraphics[width=0.8\textwidth]{linealización_decaimiento_Rh99.png}
	\caption{Linealización con logaritmos para determinar la constante de decaimiento del elemento mediante la ecuación de una recta.}
	\label{fig:img2}
\end{figure}

\begin{figure}[H]
	\centering
	\includegraphics[width=0.8\textwidth]{datos_5_half_life_Rh99.png}
	\caption{Representación gráfica de 5 vidas medias consecutivas del isótopo. Los puntos marcados corresponden a los instantes en que la actividad se reduce a la mitad respecto al valor 16.09 días anteriores.}
	\label{fig:img3}
\end{figure}

%----------------------------------------------------------------------------------
%----------------------------------------------------------------------------------

\section{Discusión de resultados}
Haciendo uso de los datos proporcionados por el notebook indicado en las instrucciones de la actividad, se dispuso de 10 mediciones de la actividad de un isótopo radiactivo inestable, distribuidas a lo largo de un intervalo de 90 días.\\

A continuación se procedió a graficar los datos de las mediciones con el fin de observar la naturaleza de los mismo. En la figura \ref{fig:img1} se presenta la actividad medida en función del tiempo. Donde la tendencia decreciente y suavemente curvada es cualitativamente compatible con el comportamiento de la ley de decaimiento exponencial, tal como se espera teóricamente para la desintegración radiactiva de un isótopo inestable.\\

Por tanto, resulta relevante determinar la constante de desintegración $\lambda$, dado que se trata de una propiedad característica del isótopo y permite relacionar experimentalmente los datos con un valor concreto de semivida tabulada.\\

Para determinar el valor de $\lambda$, se calculó el logaritmo natural de las actividades medidas, linealizando la relación exponencial. A partir de la ley de decaimiento

\begin{equation} \label{ecu.1}
	A(t)=A_0 e^{-\lambda t}
\end{equation}

se obtiene

\begin{equation} \label{ecu.2}
	\ln{A(t)}=\ln{A_0} - \lambda t 
\end{equation}

la cual tiene la forma de una recta $y=b+mx$. Por comparación, $-\lambda$ corresponde a la pendiente de la misma y puede determinarse mediante una regresión lineal por mínimos cuadrados.

En la figura \ref{fig:img2} se muestra la representación de $\ln A(t)$ en función del tiempo, junto con el ajuste lineal obtenido. Considerando el valor de $R^{2}$, se observa un excelente ajuste de los puntos con la recta, evidenciando que los datos siguen muy de cerca la ley de decaimiento exponencial y permitiendo obtener que $\lambda$ posee un valor de $(0.04308\pm0.00004) \text{ días}^{-1}$.\\

Ya con el valor de $\lambda$, resta calcular el tiempo de vida media, dado por

\begin{equation} \label{ecu.3}
	T_{1/2} = \frac{\ln{2}}{\lambda}
\end{equation}

Obteniendo, como se observa en la figura \ref{fig:img2}, que $T_{1/2}$ es igual a $16.09\text{ días}$, lo cual corresponde a la vida media de uno de los isótopos del rodio $(Rh)$, específicamente, el Rodio-99 ($^{99}Rh$) \cite{wikipedia_isotopes_rhodium}.\\

Por último, se tiene el reto propuesto en el notebook, en el apartado proyecto, donde se plantea pregunta ¿Puedes preparar una representación gráfica que muestre 5 vidas medias? 

En respuesta a esta pregunta, se construyó la gráfica de la figura \ref{fig:img3}, donde se presentan cinco puntos en los que el isótopo alcanza la mitad de actividad que tenía $16.09 \text{ días}$ antes. En la misma, imagen también se puede observar, a que día corresponde la medición y la medida de actividad correspondiente a ese día.
%----------------------------------------------------------------------------------
%----------------------------------------------------------------------------------

\section{Referencias}
\nocite{*}
\printbibliography[heading=none]

\end{document}          
